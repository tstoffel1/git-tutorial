%
% Document class
\documentclass[9pt,handout]{beamer}
%
% Theme
\usetheme[
numbering=fraction,
% background=dark,
progressbar=foot,
]{metropolis}
%
% Colors
\definecolor{murraygold}{RGB}{240,195,58}
\definecolor{murrayblue}{RGB}{2,8,69}
\setbeamercolor{normal text}{bg=white,fg=murrayblue}
\setbeamercolor{alerted text}{fg=murraygold}
\setbeamercolor{example text}{fg=murraygold}
%
% Detailed settings/packages
%
% General packages
\usepackage{textpos}
% \usepackage{fontspec}
%
% Location(s) of images
\graphicspath{{images/}}
%
% Fonts
\setsansfont[BoldFont={Fira Sans SemiBold}]{Fira Sans}
\setbeamerfont{title}{size=\huge}
\setbeamerfont{author}{size=\LARGE}
\setbeamerfont{date}{size=\normalsize}
\setbeamerfont{institute}{size=\normalsize}
\setbeamerfont{frametitle}{series=\mdseries}
%
% Margins
% \setbeamersize{text margin left=0.5cm,text margin right=0.5cm}
%
% Frame title logo
\makeatletter
\newlength{\frametitleheight}% <- NEW
\newsavebox{\beamer@titlebox}% <- NEW
\setbeamertemplate{frametitle}{%
  \ifbeamercolorempty[bg]{frametitle}{}{\nointerlineskip}%
  \@tempdima=\textwidth%
  \advance\@tempdima by\beamer@leftmargin%
  \advance\@tempdima by\beamer@rightmargin%
  \sbox{\beamer@titlebox}{% <- NEW
      \begin{beamercolorbox}[sep=0.3cm,left,wd=\the\@tempdima]{frametitle}
        \usebeamerfont{frametitle}%
        \vbox{}\vskip-1ex%
        \if@tempswa\else\csname beamer@fteleft\endcsname\fi%
        \strut\insertframetitle\strut\par%
        {%
          \ifx\insertframesubtitle\@empty%
          \else%
          {\usebeamerfont{framesubtitle}\usebeamercolor[fg]{framesubtitle}\insertframesubtitle\strut\par}%
          \fi
        }%
        \vskip-1ex%
        \if@tempswa\else\vskip-.3cm\fi% set inside beamercolorbox... evil here...
      \end{beamercolorbox}%
     }% <- NEW
     \usebox{\beamer@titlebox}% <- NEW
\settoheight{\frametitleheight}{\usebox{\beamer@titlebox}}% <- NEW
\addtolength{\frametitleheight}{\headheight}}
\makeatother
\addtobeamertemplate{frametitle}{}{%
  \begin{tikzpicture}[remember picture,overlay]
    \node[inner sep=0, outer sep=0, anchor= east,xshift=-2mm,yshift=-0.5*\frametitleheight] at (current page.north east) {\includegraphics[height=0.675cm, keepaspectratio]{murraystate-logo-shield.png}};
  \end{tikzpicture}}
%
% Progress bar width
\makeatletter
\setlength{\metropolis@progressinheadfoot@linewidth}{1pt}
\makeatother
%
% Footer
\setbeamertemplate{frame footer}{\vspace*{-5pt}\insertshorttitle\hfill\secname}
\setbeamerfont{page number in head/foot}{size=\tiny}
\setbeamercolor{footline}{fg=gray}
%
%%% Local Variables:
%%% mode: latex
%%% TeX-master: "git-tutorial"
%%% TeX-engine: xetex
%%% End:

%
% Document information
\title{Version Control}
\date{\today}
\author{Tyler D. Stoffel}
\institute{ENT 419: Senior Project I}
\titlegraphic{\vspace*{7.25cm}\includegraphics[width=4cm]{murraystate-logo-primary.png}}
%
% Start document
\begin{document}
\maketitle
%
\section{Version control}
%
\begin{frame}{Version control}
  \vspace*{\baselineskip}
  \begin{columns}
    %
    \begin{column}{0.6\linewidth}
      What is version control (VC)?
      \begin{itemize}
        \item The traditional way: New version? \\ \emph{File $\rightarrow$ Save as $\rightarrow$ myDocument\_v2.docx}
        \item The better way: Version control.
      \end{itemize}
      \onslide<2->{
        \begin{alertblock}{Version control}
          A system that records changes to files over time so that you can recall specific versions, merge changes, collaborate on developing the files, and more.
        \end{alertblock}
      }
      \vspace*{\baselineskip}
      \onslide<3->{
        Some programs that have version control built in:
        \begin{itemize}
          \item AutoCAD, OnShape, and SolidWorks
          \item Studio 5000
          \item Microsoft programs like Word, Excel, Teams (kind of)...
        \end{itemize}
      }
    \end{column}
    %
    \begin{column}{0.4\linewidth}
      \onslide<1->{
        \begin{figure}
          \centering
          \includegraphics[width=3.5cm]{version-control-comic.png}
          \caption{The traditional way.}
        \end{figure}
      }
    \end{column}
    %
  \end{columns}
\end{frame}
%
\begin{frame}{What is \texttt{git}?}
  \vspace*{\baselineskip}
  \begin{columns}
    %
    \begin{column}{0.75\linewidth}
      \texttt{git} is the most popular version control software in the world.
      \begin{itemize}
        \item Created by Linux Torvalds, creator of Linux, in 2005.
        \item Free, open source distributed version control system.
        \item Mostly for code and text, but works for other types of files too.
        \item Developed for use on the command line, i.e. \\ \texttt{\footnotesize git commit ``feat: Add third hot chocolate flavor to HMI''} \\ but many programs deal with that for you.
      \end{itemize}
      \onslide<2->{
        Some of the power of \texttt{git}:
        \begin{itemize}
          \item \texttt{git log}: Visualize the changes made to all of your files.
          \item \texttt{git diff}: See details of what was changed.
          \item \texttt{git merge}: Automatically merge different versions of files together.
        \end{itemize}
      }
    \end{column}
    %
    \begin{column}{0.25\linewidth}
      \onslide<1->{
        \begin{figure}
          \centering
          \includegraphics[width=2.25cm]{git-symbol.png}
        \end{figure}
        \vspace*{3\baselineskip}
        \begin{figure}
          \centering
          \hspace*{-\baselineskip}
          \includegraphics[width=3.5cm]{git-idea.png}
        \end{figure}
      }
    \end{column}
    %
  \end{columns}
\end{frame}
%
\begin{frame}{What is GitHub?}
  \vspace*{\baselineskip}
  \begin{columns}
    %
    \begin{column}{0.8\linewidth}
      GitHub is a cloud based \texttt{git} repository service.
      \begin{itemize}
        \item For profit, owned by Microsoft, with an annual revenue of \$1 billion.
        \item Used by 100 million developers.
        \item Usually used for public projects that everyone can use and develop themselves.
              \begin{itemize}
                \item Python, Go, Rust
                \item Linux, git itself
                \item Bitcoin
                \item Firefox, Brave Web Browser, Blender
                \item ChatGPT (soon)
              \end{itemize}
      \end{itemize}
    \end{column}
    %
    \begin{column}{0.2\linewidth}
      \onslide<1->{
        \begin{figure}
          \centering
          \includegraphics[width=1.75cm]{github-symbol.png}
        \end{figure}
      }
    \end{column}
    %
  \end{columns}
  \pause
  You've probably used GitHub before (save ZIP folder), but today we're talking about using it the powerful version control way.
\end{frame}
%
\begin{frame}
  \frametitle{Is \texttt{git} useful for EMT folks?}
  \pause
  Probably not.
  \pause
  \begin{itemize}
    \item Open source software is becoming more popular and widely used (e.g. the Click PLC).
    \item You're not software developers, but version control is a powerful tool for anything that uses scripts, code, or files.
    \item You can version control most anything that you do on a computer.
  \end{itemize}
\end{frame}
%
\begin{frame}
  \frametitle{\texttt{git} conventions}
  These are a few of the commands that are run in \texttt{git}, i.e. \texttt{git <command> ...}
  \begin{itemize}
    \item \texttt{clone}: Copy the files in the repository to your machine \emph{with} version control (not just download ZIP). \pause
    \item \texttt{commit}: A set of changes which you'd like to ``commit'' to the VC system. \pause
    \item \texttt{branch}: A set of separated commits given a name, usually meant for merging with the files on the main branch. \pause
    \item \texttt{(fork)}: Same as a branch, but distinguished because they're owned by you instead of the branch name. \pause
    \item \texttt{merge}: Tell \texttt{git} to merge a couple of branches together. \pause
    \item \texttt{diff}: Show the differences between a couple of branches or commits. \pause
    \item \texttt{log}: Show the history of changes on a branch. \pause
    \item \texttt{pull/push}: Accept or interject commits from/to the repository where the files are (GitHub). \pause
  \end{itemize}
\end{frame}
%
\section{Activity}
%
\begin{frame}
  \frametitle{Clone this repository}
  \begin{itemize}
    \item On the startup page, click \emph{Clone a Repository from the internet}.
    \item Select \emph{URL}, and enter the URL for this repository: \url{https://github.com/tstoffel1/git-tutorial.git}.
    \item Click \emph{Clone}.
  \end{itemize}
  \begin{figure}
    \centering
    \includegraphics[width=0.9\linewidth]{activity-clone}
  \end{figure}
\end{frame}
%
\end{document}
%
%%% Local Variables:
%%% mode: latex
%%% TeX-master: t
%%% End:
